\documentclass[a4paper,french]{article}
\usepackage[T1]{fontenc}
\usepackage{amsmath}
\usepackage{babel}
\usepackage[linesnumbered,ruled, lined]{algorithm2e}
\SetKwComment{Comment}{$\triangleright$\ }{}

\begin{document}
\begin{algorithm}

\caption{Division de l'espace en sous-groupes indépendants}
\Entree {\begin{itemize}
\item[] $C_{numsiege,coordX,coordY}$
\Comment*[r]{Une matrice de n lignes}
\item[] $d$
\Comment*[r]{Une distance en mètres}
\item[] $M_{i,j}$
\Comment*[r]{Une matrice de dist. eucl. vide}
\end{itemize}%
}
\Sortie{\begin{itemize}
\item[] $C*_{numsiege,coordX,coordY, \textbf{numgroupe}}$
\end{itemize}%
}
\BlankLine
Calcul des distances euclidiennes \;
\PourCh{$i \in C$}{ \:
\PourCh{$i \in C$}{ $M_{i,j}$=distance euclidienne ${i,j}$ }
}
\BlankLine
Assignation des groupes de chaises \;
G=0
\Comment*[r]{Initialiser le compteur de groupe à 0}
\PourCh{$i \in C$}{ \:
$i_{numgroupe}$=0
}
\PourCh{$n \in C$}{ \:
    \uSi{$n_{numgroupe}$=0}{
      G+=1 \;
      {$n_{numgroupe}$=G} \Comment*[r]{Assigner la chaise au groupe G}
      L=$[n]$ \Comment*[r]{Créer une liste avec l'élément n}

      \PourCh{$i \in L$}{
      \PourCh{$j \in C$} %\Comment*[r]{Examiner les voisins de la chaise choisie}
      {
      \uSi{$i!=j$ ET $M_{i,j} < d$}{
      {$j_{numgroupe}$=G} \Comment*[r]{Ajouter le voisin non-distancé au même groupe}
      L.append(j)  \Comment*[r]{L'ajouter à la liste L pour itérer à travers ses voisins}
      }
      }
      }
    }

}

\end{algorithm}

\BlankLine
\BlankLine
\BlankLine

\begin{algorithm}

\caption{Allocation des sièges}
\Entree {\begin{itemize}
\item[] $C_{numsiege,coordX,coordY}$
\Comment*[r]{Une matrice de n lignes}
\item[] $F_{numsiege,coordX,coordY,assignation}$
\Comment*[r]{Une matrice de n lignes, vide ou remplie de zéros}
\item[] $L$
\Comment*[r]{Une liste vide}
\item[] $d$
\Comment*[r]{Une distance en mètres}
\item[] $iter$
\Comment*[r]{Un nombre maximal d'itérations de stagnation}
\end{itemize}%
}
\Sortie{\begin{itemize}
\item[] $F_{numsiege,coordX,coordY,assignation}$
\Comment*[r]{Une matrice de n lignes avec l'assignation des chaises}
\end{itemize}%
}

\BlankLine
Allocation des chaises \;

compteur = 0
\Comment*[r]{Compteur des boucles while}
dernierchangement = 0
\Comment*[r]{Le numéro de boucle avec la dernière amélioration}
solutionoptimale = 0
\Comment*[r]{La solution initiale}

\Tq{$(compteur-dernierchangement) > iter$}{ 
  \Tq{$C > 0$}{ 
  Choisir $numsiege$ au hasard entre $0,n$ 
  $i$=$numsiege$
  $F_{i}=C_{i}$ \Comment*[r]{Copier la ligne i dans F à la même position}
  $F_{i,assignation}=1$  \Comment*[r]{Assigner l'occupation à 1}
   \PourCh{$j \in C$}{ 
        dist= distance euclidienne $C_{i,j}$\;
          \uSi{dist > d ET dist > 0}{
            $F_{j}=C_{j}$ \Comment*[r]{Copier la ligne j dans F à la même position}
            $F_{j,assignation}=1$   \Comment*[r]{Assigner occupation à 0 (trop proche)}
            $C_{j}$=0  \Comment*[r]{Retirer de C}
          }
        }
        $C_{i}$=0   \Comment*[r]{Retirer de C}
    }
S=somme($F_{assignation}$)\;
\uSi{S > solutionoptimale}{ \Comment*[r]{Si la solution act. est meilleure que la solution opt.}
solutionoptimale = S \Comment*[r]{Affecter sol. actuelle comme solution opt.}
dernierechangement=compteur \Comment*[r]{Enregistrer l'itération avec l'amélioration}
}

Compteur+=1

}

\end{algorithm}
\end{document}